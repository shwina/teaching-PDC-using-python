Parallel and distributed computing (PDC) terms such as multi-core processing,
GPU processing, big data, data-intensive computting, etc have become the norms
in aspects of IT across academic and industrial areas. It is critical that
college graduates are properly introduced to PDC core concepts and
technologies for their future careers. It is typical for PDC to be primarily
taught as a single course within the entire undergraduate CS curriculum while
having some PDC concepts embedded in other courses
\cite{wolffe2009teaching,butler1988introducing}.

The existing body of PDC knowledge spans across four different areas: Data
Structures and Algorithm, Software Design, Software Environments, and Hardware
\cite{brown2010strategies}. This leads to an increase in the number of
computing tools and platforms that need to be used to teach the corresponding
PDC concepts. The relevancy of including various technologies in the curriculum
is clear as PDC begins to move beyond the traditional high performance computing
concepts and into the areas of data-intensive computing, big data analytics, and
 large-scale streaming systems. The students are faced with a significant
 challenge to balance between learning both new PDC concepts and the
accompanying computing platforms.

In this paper, we describe the development of an educational approach to teach
various PDC concepts using a common computing platform based on Python. Our
approach and the accompanying teaching platform will offer the following
advantages:

\begin{itemize}
  \item The approach facilitates the delivery of most common PDC areas such as
   high performance computing, data-intensive computing, in-memory distributed
    computing, and GPU-based computing. These areas are originally designed for
    different computing platforms using different languages. The corresponding
    libraries in Python that can be leveraged to teach these areas are shown in
    Table \ref{tab:summary}.
  \item Python is a user-friendly introductory programming language with a low
    barrier of entry to new users. It is also well supported and works across
    different operating systems (Linux, MacOS, and Windows).
  \item Python libraries to support PDC are highly unified and well-defined by
    the open source community. In many cases, the performance of these libraries
    are not as good as the original. On the other hand, it is more important for
    students to spend more time understanding PDC concepts and less time on the
    technical and syntax aspects of specific tools and languages, even if they provide the optimal performance.
\end{itemize}

\begin{table*}
\caption{PDC Areas, their conventional platform/language for educational purpose,
    and the corresponding Python platforms/libraries}
\label{tab:summary}
\centering
\begin{tabular}{|l|c|c|}
\hline
    Area & Conventional platform/language & Python-based equivalent \\
\hline
    High Performance Computing & MPI (C/Fortran) & mpi4py \\
\hline
    Data-Intensive Computing and Big Data  & Hadoop MapReduce (Java) & Pure Python/MRJob \\
\hline
    In-memory Distributed Computing & Spark (Scala) & Spark/PySpark \\
\hline
    GPU-based Computing & CUDA/OpenCL (C/C++) & PyCUDA/PyOpenCL \\
\hline
\end{tabular}
\end{table*}
The remainder of this paper is organized as follows. Section \ref{sec:development} describes in detail the development of our approach and the accompanying computing platform, and how it will be used to facilitate the teaching of the four PDC areas mentioned in Table \ref{tab:summary}. Section \ref{sec:prelim} describes the preliminary responses from students as we utilize the platform to teach a one-week REU-site boot-camp workshop for data-intensive computing. Section \ref{sec:conclusion} concludes the paper and discusses future work.
