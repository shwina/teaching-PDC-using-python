%
% We have developed a Hadoop MapReduce module and taught it extensively in a course on Distributed Computing. The complex architecture of the Hadoop ecosystem and its departure from the typical high performance computing cluster design with a parallel storage file system have pushed us to experiment with a number of different approaches to teach the materials as well as to set up a Hadoop environment on which students can experiment and learn. The feedback from the students indicates a high level of satisfaction with the module, and the modifications on the myHadoop scripts allow instructors to take advantage of a centralized shared computing resource to allow students to set up individual Hadoop clusters.
%
% The Hadoop MapReduce ecosystem is a prime example of a significant area of interest which is ``continually evolving as topics mature and even newer topics appear on the scene'' \cite{nsf-tcpp:2012}. This is made evident by recent developments of the Hadoop ecosystem that have moved Hadoop beyond MapReduce's limitations in order to support additional capabilities such as cluster resource manager \cite{yarn:2013}, in-memory distributed computing \cite{spark:2013}, interactive programming \cite{tez:2013}, and distributed data store \cite{hbase:2013}. While not all of these concepts are relevant to the subject of distributed computing, they are certainly desirable skills for an employee or a graduate student to have and can be incorporated into other courses. Future work includes developing the myHadoop scripts to continue to support these new components of the Hadoop ecosystem. Efforts should also be spent on matching the materials used for teaching elements of the Hadoop ecosystem to the knowledge units within the computer science curriculum so that the students are taught the fundamental concepts in computer science but are also familiar with the frontier of computing technologies.
%
% Teaching materials from all versions of the course can be accessed at \cite{CPSC:3620}.
